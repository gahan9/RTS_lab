%\documentclass[10pt,journal,compsoc]{IEEEtran}
\documentclass[conference]{IEEEtran}
\IEEEoverridecommandlockouts
% The preceding line is only needed to identify funding in the first footnote. If that is unneeded, please comment it out.
\usepackage{cite}
\usepackage{amsmath,amssymb,amsfonts}
\usepackage{graphicx}
\usepackage{textcomp}
\usepackage{xcolor}
\def\BibTeX{{\rm B\kern-.05em{\sc i\kern-.025em b}\kern-.08em
		T\kern-.1667em\lower.7ex\hbox{E}\kern-.125emX}}
\usepackage{float}

% *** SPECIALIZED LIST PACKAGES ***
%
\usepackage{algorithmic}

% *** ALIGNMENT PACKAGES ***
%
\usepackage{array}

% *** FLOWCHART AND GRAPHS PACKAGES ***
%
\usepackage{tikz}
\usetikzlibrary{snakes,arrows,shapes, shapes.geometric, calc, automata,positioning}
\tikzstyle{startstop} = [rectangle, rounded corners, minimum width=3cm, minimum height=1cm,text centered, trapezium stretches=true, draw=black, 
%fill=red!30
]
\tikzstyle{io} = [trapezium, trapezium left angle=70, trapezium right angle=110, minimum width=3cm, trapezium stretches=true, minimum height=1cm, text centered, draw=black, 
%fill=blue!30
]
\tikzstyle{process} = [rectangle, minimum width=3cm, minimum height=1cm, text centered, draw=black, trapezium stretches=true, 
%fill=orange!30
]
\tikzstyle{decision} = [diamond, minimum width=3cm, minimum height=1cm, text centered, draw=black, trapezium stretches=true, 
%fill=green!30
]
\tikzstyle{arrow} = [thick,->,>=stealth]

% *** TABLE RELATED CHANGES ***
%
\renewcommand\arraystretch{2}
\newcolumntype{L}{>{\arraybackslash}m{4cm}}
\newcolumntype{P}{>{\arraybackslash}m{2cm}}
\newcolumntype{C}{>{\arraybackslash}m{1.2cm}}

% IEEEtran contains the IEEEeqnarray family of commands that can be used to
% generate multiline equations as well as matrices, tables, etc., of high
% quality.

\usepackage{url}
\hyphenation{op-tical net-works semi-conduc-tor}

\newcommand\rtos{Real Time Operating System}
\newcommand\rtai{Real Time Application Interface}
\newcommand\rtlinux{RTLinux}

\begin{document}
%
% paper title
\title{Comparative Analysis of RTLinux and RTAI 
%	{\footnotesize \textsuperscript{*}Note: Sub-titles are not captured in Xplore and should not be used}
%	\thanks{Identify applicable funding agency here. If none, delete this.}
}

\author{\IEEEauthorblockN{Gahan Saraiya}
	\IEEEauthorblockA{\textit{Institute of Technology} \\
		\textit{Nirma University}\\
		Ahmedabad, India \\
		18mcec10@nirmauni.ac.in}
	\and
	\IEEEauthorblockN{Rushi Trivedi}
	\IEEEauthorblockA{\textit{Institute of Technology} \\
		\textit{Nirma University}\\
		Ahmedabad, India \\
		18mcec08@nirmauni.ac.in}
}

% The paper headers
%\markboth{Consummating Research Projects using Agile Manifesto, November~2018}%
%{Shell \MakeLowercase{\textit{et al.}}: Consummating Research Projects using Agile Manifesto}

\maketitle
% make the title area
\IEEEdisplaynontitleabstractindextext
\IEEEpeerreviewmaketitle

\begin{abstract}
	This paper portrays depth analysis of \rtlinux\ and \rtai\ in context to performance and application analysis. Various performance parameters are proposed and compared in this paper.
\end{abstract}

% Note that keywords are not normally used for peerreview papers.
\begin{IEEEkeywords}
	\rtlinux, \rtai, \rtos
\end{IEEEkeywords}

\section{Introduction}
\rtai is a real-time extension for the Linux kernel, which lets users write applications with strict timing constraints for Linux. RTAI consists mainly of two parts: 
\begin{itemize}
	\item[] An Adeos-based patch to the Linux kernel which introduces a hardware abstraction layer, 
	\item[] and a broad variety of services which make lives of real-time programmers easier. 
\end{itemize}
RTAI versions over 3.0 use an Adeos kernel patch, slightly modified in the x86 architecture case, providing additional abstraction and much lessened dependencies on the "patched" operating system. Adeos is a kernel patch comprising an Interrupt Pipeline where different operating system domains register interrupt handlers. This way, RTAI can transparently take over interrupts while leaving the processing of all others to Linux. Use of Adeos also frees RTAI from patent restrictions caused by RTLinux project.

\rtlinux is a hard realtime real-time operating system (RTOS) microkernel that runs the entire Linux operating system as a fully preemptive process. The hard real-time property makes it possible to control robots, data acquisition systems, manufacturing plants, and other time-sensitive instruments and machines from \rtlinux applications.

The general idea of Real-time (RT) Linux is that a small real-time kernel runs beneath Linux, meaning that the real-time kernel has a higher priority than the Linux kernel. Real-time tasks are executed by the real-time kernel, and normal Linux programs are allowed to run when no realtime tasks have to be executed. Linux can be considered as the idle task of the real-time scheduler. When this idle task runs, it executes its own scheduler and schedules the normal Linux processes. Since the real-time kernel has a higher priority, a normal Linux process is preempted when a real-time task becomes ready to run and the real-time task is executed immediately.

\section{Performance Analysis}



% references section

%\bibliographystyle{IEEEtran}
%\bibliography{IEEEabrv,research_paper}

% that's all folks
\end{document}