

\section{Agile Principles for Research}\label{sec:agile-research-principles}
In water-fall approach one gathers requirements, analyses that requirements and design the model for implementation.
\\ In contrast, Agile Process model is based on costs/Values as single individual and interactions over processes, working on documentation, customer or user involvement and rapid response to a change with respect to plan.
These costs/Values inherited from Agile Manifesto can be observed in all Agile Methods which includes:
\begin{itemize}
	\item Scrum
	\item Extreme Programming (XP)
	\item Adaptive Software Development
	\item Dynamic System Development
	\item Crystal and Adaptive Software Development
\end{itemize}

Even in Research area this factors/values plays a vital role to maintain constant interaction between team members and instructor or supervisor for feedback and to deliver various research modules simulating the working product at the end of each iteration.
Since the on-going research would be very dynamic in nature and always encourage various changes in requirements i.e. promotes changing environment, at this point of view use of Agile Principles plays a vital role.

\begin{table}[htbp]
	\caption{Translating Agile Principles for Research}
	\newcolumntype{L}{>{\arraybackslash}m{4cm}}
	\centering
	\begin{tabular}{|L|L|}
		\hline
		\textbf{Agile Principles} & \textbf{Agile Research Principles}
		\\ \hline
		Highest priority is to satisfy the customer & Advisor should act as perfect Proxy for various customers whether the customer is Editor of Journal or Examining Committee Member
		\\ \hline
		Deliver working product frequently & Always try to organize your work  as draft of final product reports, thesis etc
		\\ \hline
		Welcome changing Environments & Always remain open for changes from your advisors/reviewers/peers.
		\\ \hline
		Various teams works together throughout the project & Update changes/modification to your advisor frequently
		\\ \hline
		Face to Face conversation to convey your message & Meet your advisor/peer assistance face to face
		\\ \hline
		Working Software is primary measures & Drafts/Thesis are primary measures
		\\ \hline
		Promotes sustainable development & Maintain constant pace
		\\ \hline
		Be attentive to technical Excellence and be good designer & Write clearly and organise your research notes
		\\ \hline
		Self-organizing team to provide best designs & Discuss methodology and results frequently
		\\ \hline
		Learn to be more effective & Learn to reflect
		\\ \hline
	\end{tabular}
\label{table:agileResearchPrinciples}
\end{table}


%\subsection{Background for the Research Problem and Methodology}

\subsection{Need for Agile Methodology in Research Projects}
	\begin{itemize}
		\item While taking into account Research-oriented projects, Agile always has proven to be excellent process
		model.
		\item For every Agile supported Research, 3 components should always be use i.e. Fundamental
		Components Of Agile:
		\begin{itemize}
			\item Transparency
			\item Inspection
			\item Adaptation
		\end{itemize}
		\item This components of Agile offer very crystal clear benefits to performance of Research.
		\item While investigating methodologies and implementing various techniques, transparency
		provides procedure for mechanism, further which it leads to adaptation.
		\item This states that researchers are benefited from Rapid feedback which is a characteristics of
		Agile process model.
		\item Agile provides quick realignment of various resources based on current results, researchers
		can save time rather than finding dead-end solutions.
		\item Transparency and inspection also insure that various priority are aligned in correct manner.
	\end{itemize}

%\subsection{}

\section{Driving Research Projects with Agile Methodology}\label{sec:drive-agile}
%\section{Iteration for research projects with Agile}
Iteration steps described in Table \ref{table:adaptAgile} are generic guideline to adapt agile methodology in research projects to grow productiveness of project outcome.
\begin{table*}[t]
	\caption{Adapting Agile Methodology in Research}
	\begin{center}
%		\newcolumntype{L}{>{\arraybackslash}m{4cm}}
		\begin{tabular}{|p{3cm}|p{4cm}|p{8cm}|}
			\hline
%			\textbf{Iteration}&\multicolumn{2}{|c|}{\textbf{Table Column Head}} \\
%			\cline{2-3} 
			\textbf{Iteration Number} & \textbf{\textit{Iteration}} & \textbf{\textit{Detail}}\\
			\hline
			Iteration 0 & Topic Identification & Identifying the topic and introduction to Agile. To come up with topic to stand out from others it is also advisable to seek for advice of supervisor/instructor
			\\ \hline
			Iteration 1 & Information Gathering & Gather detailed information about topic from various sources related to research project defined in Iteration 0. Start over from Iteration 0 to reconsider topic if available sources are not quantitatively and qualitatively enough
			\\ \hline
			Iteration 2 & CARS-checklist\cite{cars-checklist} for Information Quality & 
				\begin{itemize}
					\item \textbf{Credibility -} an authoritative source
					\item \textbf{Accuracy -} up to date source that is factual as of today
					\item \textbf{Reasonableness -} no fallacies and conflict of interest
					\item \textbf{Support -} a source which provides convincing evidence for claim
				\end{itemize}
			At the end of Iteration 2 if gathered information failed to pass CARS-checklist then start information gathering described in Iteration 1
			\\ \hline
			Iteration 3 & Initial architect & design initial architecture of system, share work among team, periodic meeting to distribute tasks effectively
			\\ \hline
			Iteration 4 & Project Management & Track assigned tasks to each individual with online project management tool
			\\ \hline
			Iteration 5 & Integration & start integrating individual module with initial designed architectural and validate the design.
			\\ \hline
			Final & Release and patent registration & register patent or publish research and release the build if integration completed at the stage where no further integration and validation require otherwise repeat from Iteration 4.
			\\ \hline
%			\multicolumn{4}{l}{$^{\mathrm{a}}$Sample of a Table footnote.}
		\end{tabular}
		\label{table:adaptAgile}
	\end{center}
\end{table*}
The iteration steps in Table \ref{table:adaptAgile} can also be revamped according to project specification.

As described in Table \ref{table:adaptAgile} Initially the person will seek for a suitable topic of research which can be applied to produce or fulfill market needs so that the research can be more applied instead of just theoretical. Until the suitable research area is found the selected topic must be evaluated with existing market and ongoing research and topic should be selected on considering its future scope. If such topic is found then first thing is to start information gathering involving team and taking help of supervisor to gather as much as relevant information. If quantitatively enough information is found then all those information must be evaluated against CARS checklist \cite{cars-checklist}. Keep all the information which passed CARS checklist and discard the remaining and if the information need to be discarded then Iteration must be repeated from Iteration 1 : Information gathering to collect more information towards the topic. 

When sufficient CARS checklist passed information is gathered then the crucial phase of architectural design comes where Research Project architecture should be defined in such a way that it can be efficiently distributed in teams and time-line. 

The task should be distributed in such a way the progress can be achieved 

\begin{figure}[htbp]
	\centering
%	\begin{tikzpicture}[node distance=2cm]
	\begin{tikzpicture}[>=latex',line join=bevel, node distance=1.8cm, scale=0.8,  every node/.style={transform shape}]
		\node (start) [startstop] {Start};
		\node (it0) [process, below of=start, text width=7cm] 
			{Iteration 0: Topic Identification};
		\node (valid-it0) [decision, below of=it0, aspect=3] {valid topic};
		\node (it1) [process, below of=valid-it0, text width=7cm] 
			{Iteration 1: Information Gathering};
		\node (it2) [process, below of=it1, text width=7cm] 
			{Iteration 2: validate CARS checklist};
		\node (valid-it2) [decision, below of=it2, aspect=3] {Passes CARS};
		\node (it3) [process, below of=valid-it2, text width=7cm] 
			{Iteration 3: Initial architect};
		\node (it4) [process, below of=it3, text width=7cm] 
			{Iteration 4: Project management};
		\node (io-integration) [io, below of=it4, yshift=0.3cm] {Individual Modules};
		\node (it5) [process, below of=io-integration, yshift=0.3cm, text width=7cm] 
			{Iteration 5: Integration};
		\node (valid-it5) [decision, below of=it5, aspect=3] {Integration complete ?};
		\node (release1) [io, below of=valid-it5] {Release build};
		\node (stop) [startstop, below of=release1, yshift=0.3cm] {Stop};
		
		\draw [arrow] (start) -- (it0);
		\draw [arrow] (it0) -- (valid-it0);
		\draw [arrow] (valid-it0) -- node[anchor=east] {yes} (it1);
		\draw [arrow] (it1) -- (it2);
		\draw [arrow] (it2) -- (valid-it2);
		\draw [arrow] (valid-it0.east) node[above right]{no} -- ++ (2.7,0) |- ($(start)!0.5!(it0)$);
		\draw [arrow] (valid-it2) -- node[anchor=east] {yes} (it3);
		\draw [arrow] (valid-it2.east) node[above right]{no} -- ++ (2.7,0) |- (it1);
		\draw [arrow] (it3) -- (it4);
		\draw [arrow] (it4) -- (io-integration);
		\draw [arrow] (io-integration) -- (it5);
		\draw [arrow] (it5) -- (valid-it5);
		\draw [arrow] (valid-it5) -- node[anchor=east] {yes} (release1);
		\draw [arrow] (valid-it5.east) node[above right]{no} -- ++ (2,0) |- (it4);
		\draw [arrow] (release1) -- (stop);
%		\draw [arrow] (valid-it2.east) node[above right]{no} -- ++ ([xshift=1.5cm]it1) |- ($(valid-it0)!0.5!(it1)$);
	\end{tikzpicture}
	\caption{Flowchart to follow iterations described in Table \ref{table:adaptAgile}}
\end{figure}
%\hfill mds
% 
\subsection{Benefits of Implementation}
\begin{itemize}
	\item Efficiently handling research project
	\item Research Progress can be tracked easily
	\item Overhead on one person can be avoided
	\item Distribution of research task among team can be divided efficiently with hierarchy i.e. one team can work on theoretical aspects of research at initial stage and while another theories are concluded another team can be assigned to evaluated those theories to increase progress rate.
\end{itemize}

\subsection{Challenges towards Implementation}
\begin{itemize}
	% TODO: find scrum master related paper related to research and cite here
	\item Scrum master has a responsible role at early stage
	\item Topic Identification should not be taken lightly and need to perform accurate market survey for future trends
\end{itemize}


\section{Performance Analysis}\label{sec:analysis}
Since the goal of this exploration is to examine the appropriateness of Agile techniques to investigate extends as specified before, the relationship among the factors were distinguished utilizing SPSS programming. At that point the investigation is reached out to discover the how the Agile strategies fluctuate as indicated by the kind of the undertaking. In the Table \ref{table:scrum} relationship between the task type and the propriety of scrum is broke down. Scrum is the broadly utilized Agile procedure in the greater part of the product organizations these days \cite{5765710}. Scrum is fitting for the customer based undertakings with past experience. Following table breaks down whether the scrum is suitable for the examination based activities.

\begin{table}[htbp]
	\caption{}
	\begin{center}
		\renewcommand\arraystretch{2}
		\begin{tabular}{|c|c|c|}
			\hline
			\textbf{}&\multicolumn{2}{c|}{\textbf{Is Scrum Appropriate?}}
			\\ \cline{2-3}
			\textbf{Type} & Yes (\%) & No (\%)
			\\ \hline
			Research Based & 35.3 & 13.5
			\\ \hline
			Client Based & 44.4 & 6.8
			\\ \hline
		\end{tabular}
	\end{center}
Pearson Chi Square Value  4.293 
\\ Significance Value 0.038
\label{table:scrum}
\end{table}

Agreeing the appropriate responses acquired from the study it tends to be seen  that around 35 percent trust that scrum is fitting for the examine based undertakings. The way that scrum is proper for the client based projects can be seen with the above table since in excess of 44 percent trust it is proper for customer based projects. Chi square trial of independency furnish proof with 95\% dimension of certainty that disposition on fittingness depends on the project type. It very well may be finished up with the above qualities that scrum is additionally fitting for the exploration based tasks. Outline of the reactions that were gotten whether the short emphases terrible for research based tasks are given in the Table \ref{table:shortIteration}.

Summary of the responses that were obtained whether the short iterations bad for research based projects are given in the Table \ref{table:shortIteration}.

\begin{table}[htbp]
	\caption{}
	\begin{center}
		\renewcommand\arraystretch{2}
		\begin{tabular}{|c|c|}
			\hline
%			\textbf{}&\multicolumn{2}{|c|}{\textbf{Is Scrum Appropriate?}}
%			\\ \cline{2-3}
			\textbf{Type} & \textbf{Short Iteration Bad (\%)}
			\\ \hline
			Research Based (\%) & 61.5
			\\ \hline
			Client Based (\%) & 38.5
			\\ \hline
		\end{tabular}
	\end{center}
%Pearson Chi Square Value  4.293 
%\\ Significance Value 0.038
\label{table:shortIteration}
\end{table}

It very well may be seen from the Table \ref{table:shortIteration} that state of mind whether the short cycles are terrible for research based activities is high contrasted with the customer based undertakings. With Binomial test it very well may be finished up with 95\% dimension of certainty that more than 61\% don't propose that short emphases are terrible for the exploration based undertakings. At that point the examination concentrated on reliance between the short emphases and little groups as for the venture type. Table \ref{table:smallTeamIteration} is an outline of the reactions that were accumulated with the chi-square test. 

\begin{table}[htbp]
	\caption{Short Iteration Impact for small team}
	\begin{center}
		\renewcommand\arraystretch{2}
		\begin{tabular}{|c|c|c|c|}
			\hline
			\textbf{}& \textbf{} &\multicolumn{2}{c|}{\textbf{Small Team (\%)}}
			\\ \cline{3-4}
			\textbf{Type} & \textbf{Impact} & Yes (\%) & No (\%)
			\\ \hline
			Research & Bad & 15.4 & 9.2
			\\ \cline{2-4}
			Based & Good& 70.8 & 4.6
			\\ \hline
			Client & Bad & 7.4 & 7.4
			\\ \cline{2-4}
			Based & Good& 70.6 & 14.7
			\\ \hline
		\end{tabular}
	\end{center}
	Pearson Chi Square Value  9.955
	\\ Significance Value 0.002
\label{table:smallTeamIteration}
\end{table}


From the Table \ref{table:smallTeamIteration} it very well may be reasoned that for both research based and customer based activities there is a reliance between little groups and short emphases. From the table it very well may be moreover seen that respondents who are stating that little groups are great and short emphases are awful is less in customer based ventures regarding the examination based tasks. This reasons that for customer based ventures little groups isn't great where for research based tasks little groups appears to be great. Visit advance checking is a standard of Agile techniques \cite{agilemanifesto}. Table \ref{table:progressMonitoring} is a rundown of inclination for continuous advance checking regarding the short emphases. This table breaks down whether the incessant advancement checking is compelling for short emphases.

\begin{table}[htbp]
	\caption{Short Iteration Impact on continuous advance checking}
	\begin{center}
		\renewcommand\arraystretch{2}
		\begin{tabular}{|c|c|c|}
			\hline
			\textbf{}&\multicolumn{2}{c|}{\textbf{Frequent Monitoring Progress (\%)}}
			\\ \cline{2-3}
			\textbf{Impact} & Yes (\%) & No (\%)
			\\ \hline
			Bad & 13.5 & 6.0
			\\ \hline
			Good & 72.9 & 7.5
			\\ \hline
		\end{tabular}
	\end{center}
	Pearson Chi Square Value  8.204
	\\ Significance Value 0.004
	\label{table:progressMonitoring}
\end{table}

From the Chi square test it very well may be closed with 95\% dimension of certainty that visit advance observing relies upon short cycles. With the reality it tends to be seen over 72\% feel that advancement ought to be checked in short cycle regardless of the task type. This recommends visit advance observing ought to be done not just for customer based extends yet in addition for the examination based ventures. 
Every day standup meeting is another Agile rule \cite{agilemanifesto}. Be that as it may, as students it isn't down to earth to have day by day gatherings.  In this manner we propose having week after week standup gatherings. As made reference to before a large portion of the respondents did not have a opportunity to meet their boss regularly. Table \ref{table:weeklyStandup} is a synopsis of results that were acquired.

\begin{table}[htbp]
	\caption{Impact of Embracing Change}
	\begin{center}
		\renewcommand\arraystretch{2}
		\begin{tabular}{|c|c|c|}
			\hline
			\textbf{}&\multicolumn{2}{c|}{\textbf{Periodic Standup Meeting(\%)}}
			\\ \cline{2-3}
			\textbf{Type} & Yes (\%) & No (\%)
			\\ \hline
			Bad & 13.5 & 6.0
			\\ \hline
			Good & 72.9 & 7.5
			\\ \hline
		\end{tabular}
	\end{center}
	Pearson Chi Square Value 6.625
	\\Significance Value 0.010
	\label{table:weeklyStandup}
\end{table}

Respondents who are stating yes to week by week stand up gatherings what's more, who think grasping change prerequisites is better, is more than 81\%. Additionally with the Chi square importance esteem it can be finished up with the 95\% dimension of certainty that grasp change prerequisites relies upon week after week standup gatherings. Thusly a proposal can be given that for look into based ventures that there is no requirement for day by day standup gatherings and it is reasonable to have week by week standup gatherings so they can grasp change prerequisites. With the end goal to utilize dexterity if there is any difference in prerequisites, correspondence among the group is critical. Significance of correspondence is broke down as for the group estimate. Table 4.6 is an delineation of the significance of correspondence among groups.


\begin{table}[htbp]
	\caption{Communication Impact on Research based Projects}
	\begin{center}
		\renewcommand\arraystretch{2}
		\begin{tabular}{|c|c|c|}
			\hline
			\textbf{Communication} & \multicolumn{2}{c|}{\textbf{Small Team (\%)}}
			\\ \cline{2-3}
			\textbf{among Team} & Yes & No 
			\\ \hline
			Yes & 84.6 & 13.8
			\\ \hline
			No & 1.5 & 0.0
			\\ \hline
		\end{tabular}
	\end{center}
	Pearson Chi Square Value 0.163
	\\ Significance Value 0.686
	\caption{Communication Impact on Client based Projects}
	\begin{center}
		\renewcommand\arraystretch{2}
		\begin{tabular}{|c|c|c|}
			\hline
			\textbf{Communication} & \multicolumn{2}{c|}{\textbf{Small Team (\%)}}
			\\ \cline{2-3}
			\textbf{among Team} & Yes & No 
			\\ \hline
			Yes & 75 & 13.8
			\\ \hline
			No & 1.5 & 0.0
			\\ \hline
		\end{tabular}
	\end{center}
	Pearson Chi Square Value 4.519
	\\ Significance Value 0.034
	\label{table:communication}
\end{table}

It very well may be seen that little groups does not rely upon correspondence among the group with 95\% dimension of certainty. Be that as it may, for the customer based undertakings there is a reliance between the correspondence and little groups. This finishes up that independent of the group estimate, correspondence is critical for research based activities. Additionally it is apparent from the Table \ref{table:communication} correspondence among the group is significantly more required for the exploration based activities than the customer based undertakings. Table \ref{table:embraceChangeRequirements} is a depiction of how the change prerequisites rely upon the short cycles.

\begin{table}[htbp]
	\caption{Requirements for Embracing Change in Short Iteration}
	\begin{center}
		\renewcommand\arraystretch{2}
		\newcolumntype{P}{>{\centering\arraybackslash}m{2cm}}
		\begin{tabular}{|c|c|P|c|}
			\hline
			\textbf{}& \textbf{} &\multicolumn{2}{c|}{\textbf{Embrace Change  Requirements(\%)}}
			\\ \cline{3-4}
			\textbf{Type} & \textbf{Impact} & Good (\%) & Bad (\%)
			\\ \hline
			Research & Good & 66.2 & 9.2
			\\ \cline{2-4}
			Based & Bad & 20.0 & 4.6
			\\ \hline
			Client & Good & 70.6 & 14.7
			\\ \cline{2-4}
			Based & Bad & 8.8 & 5.9
			\\ \hline
		\end{tabular}
	\end{center}
	Pearson Chi Square Value (Research Based Projects) 2.702 
	\\ Significance Value (Research Based Projects) 0.100
	\\ Pearson Chi Square Value (Client Based Projects) 2.702
	\\ Significance Value (Client Based Projects) 0.100
	\label{table:embraceChangeRequirements}
\end{table}

As per the Table \ref{table:embraceChangeRequirements} it is obvious with 95\% dimension of certainty that for research based activities there is no reliance between short emphases and grasp change necessities yet for the customer based tasks there is a reliance between grasp change prerequisites. It is a well established actuality that exploration constructed ventures center in light of one goal while customer constructed ventures not center in light of whole undertaking in the given time. Customer constructed ventures center in light of accomplishing targets for cycle.