
%\IEEEraisesectionheading{\section{Introduction}\label{sec:introduction}}
\section{Introduction}\label{sec:introduction}
\IEEEPARstart{T}{here} are thousands of system development methodologies varied by various paradigm from the object oriented paradigm to web development and to agile manifesto \cite{Strode2003}. Among such thousands of system development methodologies, it is a tough task for researchers to select a appropriate methodology which can be well organized. In the recent past, Agile has begun to play a major role in the software industry \cite{5765710} and various studies demonstrate that agile development is becoming mature, more effective and has a wide acceptance when compared with other recent trends \cite{ExecutiveBrief.2011} \cite{El-Abbassy2010}. This emerging need had influenced universities and other institutions to start teaching courses on agile practices for postgraduates \cite{Mann2006}. This paper would elaborate on a research study carried out to observe how effective applying agile methodologies like Scrum for postgraduate research projects. The appropriateness is the important factor which is mostly focused and the main objective is to give some recommendations for the Agile main principles to map it with research projects. Even though Extreme Programming is the most widely used agile process \cite{Pressman:2009:SEP:1593949}, a recent survey reveals that Scrum \cite{5765710} \cite{Schwaber:2001:ASD:559553} \cite{8442128} is extensively used in the industry. Hence, for practicing professionalism, it was recommended to use Scrum \cite{8442128} \cite{agilemanifesto} amongst other agile approaches like Extreme Programming \cite{agilemanifesto} or Adaptive Software Development \cite{Schwaber:2001:ASD:559553} \cite{agilemanifesto}. 