\section{\rtsS Model}
A \rtsS model can be defined with below parameters:
\begin{itemize}
    \item Workload model (task graph) is about what behaviors system will be asked to perform
        \\ partially ordered set of tasks with timing information for each task where ordering represents dependency for the task
    \item Resource model (Capacity graph) is about tools and capabilities system applies
        \\ contains all of the passive resources that the system must use along with their acceptable patterns of use
    system resources are divided in to:
    \begin{itemize}
        \item Processors - active resources such as computers, database servers, etc.
        \item Resources - passive resources such as memory, semaphores, locks, other data.
    \end{itemize}
    \item Scheduling algorithm is about activity coordinated by system
        \\ considers the dependencies and interferences among the tasks
    \item Resource management algorithm
    \\ Activity coordinated by system
\end{itemize}

Each job in \rtsS can be defined as a unit work and set of such jobs together called as \textit{task set} achieves desired activity. Each and every job of task set runs on a processor and depends on some resources.
\rtsS job can be characterizes as:
\begin{itemize}
    \item temporal parameters
    \item functional parameters
    \item resource parameters
    \item interconnection parameters
\end{itemize}
Of course, the very important information for the real-time analysis are the timing prediction parameters. The execution time of jobs has many uncertainties due to following variable parameters:
\begin{itemize}
    \item execution - caches, pipelines
    \item internal - branches, loops
    \item external - preemption, interference
\end{itemize}
All of these issues lead to difficulties in making a priori decisions about how much time to allow, and promote the use of simulation models to help.

\section{Issues for Programming Environments}
For many years, job of real time system programmer is to write a program with an execution time which is lesser than the time constrained allowed. To predict the precise execution time of a program is very difficult hence it usually takes multiple rounds of trail and error method to build a reliable \rts. Also to port existing real-time software to a new configuration rather than building it from scratch takes the same level of effort which is not desirable since the requirements for real-time software are stronger than ever, because of ubiquity of computer systems in all application areas. And not only this but the next-gen real-time applications have more strict timing constraints.

To cope with such issues one needs to think of out-of-the box solution. Such an approach could be to make the system very flexible such that it can meet wide range of timing constraint under various system configurations. Thus the approach focuses on scalability for wide range of timing constraints rather than depending on precise execution time of computation.
In many hard real-time systems, the requirement for functional correctness is not as strict as the requirement for temporal correctness. For these applications, the flexible performance approach is thus preferred\cite{Lin1995}.

To design and develop the next generation \rtsS using approach described above, a programming language which has constructs for right away expressing timing constraints is required. The programmer defines the temporal constraints for the computation and also provides all possible set of codes to be executed. The runtime system along with the compiler and the scheduler must decide and/or generate the code for the execution so that the constraints are met.

Many language features are desirable for real-time systems programming. In fact, almost all features desirable in a "good" conventional (i.e., non-real-time) programming language can be considered as essential for real-time systems programming. For example, five requirements have been suggested for real-time software in \cite{Stoyenko1993}: predictability, reliability, tasking, modularity, and maintainability. Other then predictability all remaining four are always highly desirable in every system. 

\subsection{Loop size, timer granularity, multi-programming, etc}
Programmer needs to take account of loops size, timer granularity, multi-programming, imprecise timer, sleep() etc. to avoid potential timing hazards

\subsection{Sequential programs, parallel programs, timely programs}

\subsection{Client-server priority assignments - priority inversion}

\subsection{Verification, analysis, and testing}